\chapter{Seasonality}

\section{The Nature and Source of Seasonality}
A seasonal pattern is one that repeats itself every year.\\
Seasonality arises from links of technologies, preferences, and institutions to the calendar.

\section{Dealing with Seasonality}
One way to deal with seasonality in a series is simply to remove it and then to model and forecast the \textbf{\color{red}seasonally adjusted} or \textbf{\color{red}deseasonalised} \textbf{\color{blue}time series}.\\
This strategy is perhaps appropriate in certain situations, such as when interest centers explicitly on forecasting \textbf{\color{ForestGreen}non-seasonal fluctuations}.


\section{Seasonal Model}
A key technique for modelling seasonality is \textbf{\color{blue}regression on dummy variables}.\\
The pure seasonal model is:
\begin{eqnarray}
	y_{t} &=& \sum_{i = 1}^{s}\gamma_{i}D_{it} + \epsilon_{t}
\end{eqnarray}
Effectively, we are just regressing on an intercept, but we allow for a different intercept in each season. Those different intercepts, the $\gamma$'s are called "seasonal factors"; they summarise the seasonal patterns over the year.\\
\textbf{\color{blue}Instead of including a full set of $s$ seasonal dummies, we can include and $s - 1$ seasonal dummies and intercept}.\\
\textbf{\color{ForestGreen}In no case, however, should we include $s$ seasonal dummies and an intercept}.

\section{Calendar Effects}
\begin{itemize}
	\item \textbf{\color{blue}holiday variation} - refers to the fact that some holiday's dates change over time
	\item \textbf{\color{blue}trading day variation} - refers to the fact that different months contain different numbers of trading or business days.
\end{itemize}