\chapter{Bernoulli Distribution}
The Bernoulli distribution is a discrete distribution having two possible outcomes labelled by $n = 0$ and $n = 1$. In which
\begin{itemize}
	\item $n = 1$ ("success") occurs with probability $p$
	\item and $n = 0$ ("failure") occurs with probability $q = 1 - p$
\end{itemize}
where $0 < p < 1$.

\section{Probability Density Function}
It therefore has probability density function
\begin{eqnarray}
	P(n) &=& \begin{cases}
		1 - p, & \text{for $n = 0$}\\
		p, & \text{for $n = 1$}
	\end{cases}
\end{eqnarray}

\section{Cumulative Distribution Function}
\begin{eqnarray}
 F(n) &=& \begin{cases}
 	1 - p, &\text{for $n = 0$}\\
 	1, & \text{for $n = 1$}
 \end{cases}
\end{eqnarray}

\section{Mean}
Let $X$ be a discrete random variable with the Bernoulli distribution with parameter $p$.\\
Then the expectation of $X$ is given by:
\begin{eqnarray}
	\mathbb{E}\big(X\big) &=& p
\end{eqnarray}

\paragraph{Proof}
From the definition of expectation:
\begin{eqnarray}
	\mathbb{E}\big(X\big) &=& \sum_{x\in\Omega_{X}}x Pr\big(X = x\big)
\end{eqnarray}
By definition of Bernoulli  distribution:
\begin{eqnarray}
	\mathbb{E}\big(X\big) &=& 1 x p + 0 x (1 - p) = p
\end{eqnarray}

\section{Variance}
The variance of $X$ is given by:
\begin{eqnarray}
	\sigma^{2}\big(X\big) &=& p\big(1 - p\big)
\end{eqnarray}

\paragraph{Proof}
From the definition of variance:
\begin{eqnarray}
	\nonumber
	\sigma^{2}\big(X\big) &=& \mathbb{E}\bigg(\Big(X - \mathbb{E}\big(X\big)\Big)^{2}\bigg)
\end{eqnarray}
From the expectation of Bernoulli distribution, we have $\mathbb{E}\big(X\big) = p$
\begin{eqnarray}
	\nonumber
	\sigma^{2}\big(X\big) &=& \mathbb{E}\bigg(\Big(X - \mathbb{E}\big(X\big)\Big)^{2}\bigg)\\
	\nonumber
	&=& \big(1 - p\big)^{2} p + \big(0 - p\big)^{2}\big(1 - p\big)\\
	\nonumber
	&=& p - 2p^{2} + p^{3} + p^{2} - p^{3}\\
	&=& p(1 - p)
\end{eqnarray}
The derivation from "Variance equals expectation of square minus square of expectation" is a bit simpler.


\section{Moments - Summary}
\begin{eqnarray}
	\mu &=& p\\
	\sigma^{2} &=& p(1 - p)\\
	\gamma_{1} &=& \frac{1 - 2p}{\sqrt{p(1 - p)}}\\
	\gamma_{2} &=& \frac{6p^{2} - 6p + 1}{p(1 - p)}
\end{eqnarray}
