\chapter{Poisson Distribution}
Poisson distribution is a discrete probability distribution that defines the probability of a given number of events occurring in a fixed interval of time or space if these events occur with a known constant rate and independently of the time since the last event. The Poisson distribution can also be used for the number of events is other specified intervals such as distance, area or volume.\\
The average number of events in an interval is designed as $\lambda$. Lambda is the event rate, also called the rate parameter. The probability of observing $k$ events in an interval is given by the equation:
\begin{eqnarray}
	P(\text{k events in interval}) &=& \frac{\lambda^{k}}{k!}e^{-\lambda}
\end{eqnarray}
This equation can be adapted if, instead of the average number of events $\lambda$, we are given a time rate $r$ for the events to happen. Then $\lambda = rt$ (with $r$ in units of $1 / time$), and
\begin{eqnarray}
	P(\text{k events in interval } t) &=& e^{-rt}\frac{(rt)^{k}}{k!}
\end{eqnarray}
The Poisson distribution could be obtained as the limit from the binomial distribution. The probability of obtaining exactly $n$ successes in $N$ trials is:
\begin{eqnarray}
	P_{p}\big(n\bigr\rvert N\big) &=& \frac{N!}{n! (N - n)!}p^{n}(1 - p)^{N - n}
\end{eqnarray}
Viewing the distribution as a function of the expected number of successes:
\begin{eqnarray}
	\lambda &=& Np
\end{eqnarray}
instead of the sample size $N$ for fixed $p$, the equation above then becomes:
\begin{eqnarray}
	P_{\lambda}\big(n\bigr\rvert N\big) &=& \frac{N!}{n! (N - n)!}\frac{\lambda}{N}^{n}\Big(1 - \frac{\lambda}{N}\Big)^{N - n}
\end{eqnarray}
Letting the sample size $N$ become large, the distribution then approaches:
\begin{eqnarray}
	\nonumber
	P_{\lambda}(n) &=& \lim_{N \rightarrow\infty}P_{p}\\
	\nonumber
	&=& \lim_{N \rightarrow \infty}\frac{N(N - 1)...(N - n + 1)}{n!}\frac{\lambda^{n}}{N^{n}}\bigg(1 - \frac{\lambda}{N}\bigg)^{N}\bigg(1 - \frac{\lambda}{N}\bigg)^{-n}\\
	\nonumber
	&=& \lim_{N \rightarrow \infty}\frac{N(N - 1)...(N - n + 1)}{N^{n}}\frac{\lambda^{n}}{n!}\bigg(1 - \frac{\lambda}{N}\bigg)^{N}\bigg(1 - \frac{\lambda}{N}\bigg)^{-n}\\
	\nonumber
	&=& 1\cdot \frac{\lambda^{n}}{n!} \cdot e^{-\lambda} \cdot 1\\
	&=& \frac{\lambda^{n}}{n!}e^{-\lambda}	
\end{eqnarray}
Note that the sample size N has completely dropped out of the probability function, which has the same functional form for all values of $\lambda$.\\
In this case, $n$ is the number of events that occur in an interval, and $\lambda$ is the expected number of events in the interval.\\
If the rate at which events occur over time is constant, and the probability of any one event occurring is independent of all other events, then we say that the events follow a Possion process, where
\begin{eqnarray}
	P[X = n] &=& \frac{(\lambda t)^{n}}{n!}e^{-\lambda t}
\end{eqnarray}
where $t$ is the time elapsed. In other words, the expected number of events before time $t$ is equal to $\lambda t$.

\section{Properties}
\begin{itemize}
	\item the Poisson distribution is normalised so that the sum of probabilities equals 1 since:
	\begin{eqnarray}
		\sum_{n = 0}^{\infty}P_{\lambda}(n) = e^{-\lambda}\sum_{n = 0}^{\infty}\frac{\lambda^{n}}{n!} = e^{-\lambda}e^{\lambda} = 1
	\end{eqnarray}
\end{itemize}

\section{Mean}
The expectation of $X$ is given by:
\begin{eqnarray}
	\mathbb{E}\big(X\big) &=& \lambda
\end{eqnarray}

\paragraph{Proof:}
From the definition of expectation
\begin{eqnarray}
	\mathbb{E}\big(X\big) &=& \sum_{x \in\Omega_{X}}x\cdot Pr\big(X = x\big)
\end{eqnarray}
By definition of Poisson distribution
\begin{eqnarray}
	\nonumber
	\mathbb{E}\big(X\big) &=& \sum_{k \geq 0}k\frac{1}{k!}\lambda^{k}e^{-\lambda}\\
	\nonumber
	&=& \lambda e^{-\lambda} \sum_{k \geq 1}\frac{1}{\big(k - 1\big)!}\lambda^{k - 1} \quad\text{as the } k = 0 \text{ term vanishes}\\
	\nonumber
	&=& \lambda e^{-\lambda} \sum_{j \geq 0}\frac{1}{j!}\lambda^{j} \qquad\qquad\quad\text{putting } j = k - 1\\
	\nonumber
	&=& \lambda e^{-\lambda} e^{\lambda} \qquad\qquad\qquad\qquad\text{Taylor series expansion for exponential function}\\
	\nonumber
	&=& \lambda
\end{eqnarray}

\section{Variance}
\begin{eqnarray}
	\sigma^{2} &=& \lambda
\end{eqnarray}

\paragraph{Proof:}
From the definition of \textbf{\color{blue} Variance as Expectation of Square Minus Square of Expectation}:
\begin{eqnarray}
	var\big(X\big) &=& \mathbb{E}\big(X^{2}\big) - \Big(\mathbb{E}\big(X\big)\Big)^{2}
\end{eqnarray}
From the \textbf{\color{blue}expectation of function of discrete random variable}:
\begin{eqnarray}
	\mathbb{E}\big(X^{2}\big) &=& \sum_{x \in \Omega_{X}}x^{2}Pr\big(X = x\big)
\end{eqnarray}
So:
\begin{eqnarray}
	\nonumber
	\mathbb{E}\big(X^{2}\big) &=& \sum_{k \geq 0}k^{2}\frac{1}{k}\lambda^{k}e^{-\lambda} \qquad\qquad\qquad\qquad\text{definition of Poisson Distribution}\\
	\nonumber
	&=& \lambda e^{-\lambda}\sum_{k \geq 1}k\frac{1}{\big(k - 1\big)!}\lambda^{k - 1} \qquad\qquad\text{note change of limit: term is zero when k = 0}\\
	\nonumber
	&=& \lambda e^{-\lambda}\Bigg(\sum_{k \geq 1}(k - 1)\frac{1}{\big(k - 1\big)!}\lambda^{k - 1} + \sum_{k \geq 1}\frac{1}{\big(k - 1\big)!}\lambda^{k - 1}\Bigg)\\
	\nonumber
	&=& \lambda e^{-\lambda}\Bigg(\lambda\sum_{k \geq 2}\frac{1}{\big(k - 2\big)!}\lambda^{k - 2} + \sum_{k \geq 1}\frac{1}{\big(k - 1\big)!}\lambda^{k - 1}\Bigg)\\
	\nonumber
	&=& \lambda e^{-\lambda}\Bigg(\lambda\sum_{i \geq 0}\frac{1}{i!}\lambda^{i} + \sum_{j \geq 0}\frac{1}{j!}\lambda^{j}\Bigg) \quad\quad\text{putting } i = k - 2\text{, } j = k - 1\\
	\nonumber
	&=& \lambda e^{-\lambda}\bigg(\lambda e^{\lambda} + e^{\lambda}\bigg) \qquad\qquad\qquad\quad\text{Taylor series expansion for exponential function}\\
	&=& \lambda\big(\lambda + 1\big) = \lambda^{2} + \lambda
\end{eqnarray}
Then:
\begin{eqnarray}
	\nonumber
	var(X) &=& \mathbb{E}\bigg(X^{2}\bigg) - \bigg(\mathbb{E}\big(X\big)\bigg)^{2}\\
	&=& \lambda^{2} + \lambda - \lambda^{2} = \lambda
\end{eqnarray}

\section{Moments - Summary}
\begin{eqnarray}
	\mu &=& \lambda\\
	\sigma^{2} &=& \lambda\\
	\gamma_{1} &=& \sqrt{\lambda}\\
	\gamma_{2} &=& \frac{1}{\lambda}
\end{eqnarray}
