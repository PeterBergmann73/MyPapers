\chapter{Uniform Distribution}
Continuous uniform or rectangular distribution is a family of symmetric probability distributions, such that for each member of the family, all intervals of the same length on the distribution support are equally probable.\\
The support is defined by two parameters, $a$ and $b$, which are its minimum and maximum values.

\section{Probability Density Function}
\begin{eqnarray}
	f(x) &=& \begin{cases}
	\frac{1}{b - a} & \text{for } a \leq x \leq b\\
	0, & \text{for } x < a \text{ or } x > b
	\end{cases}
\end{eqnarray}
In terms of mean $\mu$ and variance $\sigma$ the p.d.f. may be written as:
\begin{eqnarray}
	f(x) &=& \begin{cases}
		\frac{1}{2\sigma\sqrt{3}} & \text{for } -\sigma\sqrt{3} \leq x - \mu\leq \sigma\sqrt{3}\\
		0 & \text{otherwise}
	\end{cases}
\end{eqnarray}
These can be written in terms of the Heaviside step function $H(x)$ as:
\begin{eqnarray}
	f(x) &=& \frac{H(x - a) - H(x - b)}{b - a}
\end{eqnarray}

\section{Cumulative Distribution Function}
\begin{eqnarray}
	F(x) &=& \begin{cases}
		0 & \text{for } x < a\\
		\frac{x - a}{x - b} & \text{for } a \leq x \leq b\\
		1 & \text{for } x > b
	\end{cases}
\end{eqnarray}
Its inverse is:
\begin{eqnarray}
	F^{-1}(p) &=& a + p(b - a) \text{ for } 0 < p < 1
\end{eqnarray}
In mean and variance notation, the cumulative distribution function is:
\begin{eqnarray}
F(x) &=& \begin{cases}
0 & \text{for } x < \mu -\sigma\sqrt{3}\\
\frac{1}{2}\Big(\frac{x - \mu}{\sigma\sqrt{3}} + 1\Big) & \text{for } \mu -\sigma\sqrt{3} \leq x \leq \mu +\sigma\sqrt{3}\\
1 & \text{for } x > \mu +\sigma\sqrt{3}
\end{cases}
\end{eqnarray}
And its inverse is:
\begin{eqnarray}
	F-1(p) &=& \sigma\sqrt{3}(2p - a) + \mu \text{ for } 0 \leq p \leq 1
\end{eqnarray}
In the terms of the Heaviside step function $H(x)$:
\begin{eqnarray}
	F(x) &=& \frac{(x - a)H(x - a) - (x - b)H(x - b)}{b - a}
\end{eqnarray}


\section{Generating Functions}

\paragraph{Moment Generating Function}
The moment generating function is:
\begin{eqnarray}
	\nonumber
	M_{x} &=& \mathbb{E}\Big(e^{tx}\Big) = \int_{a}^{b}\frac{e^{tx}}{b - a}dx = \frac{e^{tx}}{t(b - a)}\biggr\rvert_{a}^{b}\\
	&=& \begin{cases}
	\frac{e^{tb} - e^{ta}}{t(b - a)} & \text{for } t \neq 0\\
	1 & \text{for } t = 0
	\end{cases} 
\end{eqnarray}


\section{Moments}
The moment-generating function is not differentiable at zero, but the moments
can be calculated by differentiating and then taking $\lim_{t \rightarrow 0}$.
However, it is easier to calculate the moments from the definition directly.

\paragraph{Row Moments}
\begin{eqnarray}
	\nonumber
	\mu_{n} &=& \int_{-\infty}^{\infty}\frac{H(x - a) - H(x - b)}{b - a}x^{n}dx\\
	\nonumber
	&=& \int_{a}^{b}\frac{x^{n}}{b - a}dx\\
	&=& \frac{b^{n + 1} - a^{n +  1}}{(n + 1)(b - a)}
\end{eqnarray}
The first few are:
\begin{eqnarray}
	\mu_{1} &=& \frac{1}{2}(a + b)\\
	\mu_{2} &=& \frac{1}{3}(a^{2} + ab + b^{2})\\
	\mu_{3} &=& \frac{1}{4}(a + b)(a^{2} + b{2})\\
	\mu_{4} &=& \frac{1}{5}(a^{4} + a^{3}b + a^{2}b^{2} + ab^{3} + b^{4})
\end{eqnarray}

\paragraph{Central Moments}
The central moments are given analytically by:
\begin{eqnarray}
	\nonumber
	\mu_{n} &=& \int_{-\infty}^{\infty}\frac{H(x - a) - H(x - b)}{b - a}\left[x - \frac{1}{2}(a + b)\right]^{n}dx\\
	\nonumber
	&=& \int_{a}^{b}\frac{\Big[x - \frac{1}{2}(a + b)\Big]^{n}}{b - a}ds\\
	&=& \frac{\Big(a - b\Big)^{n} + \Big(b - a\Big)^{n}}{2^{n + 1}\Big(n + 1\Big)}
\end{eqnarray}
The first few are:
\begin{eqnarray}
	\mu_{1} &=& 0\\
	\mu_{2} &=& \frac{1}{12}(b - a)^{2}\\
	\mu_{3} &=& 0\\
	\mu_{4} &=& \frac{1}{80}(b - a)^{4}
\end{eqnarray}


\section{Mean, Variance, Skew, Kurtosis Excess}
Mean (the first raw moment) is:
\begin{eqnarray}
	\mu &=& \frac{1}{2}(a + b)
\end{eqnarray}
Variance (the second central moment) is:
\begin{eqnarray}
	\sigma^{2} &=& \frac{1}{12}(b - a)^{2}
\end{eqnarray}
Skew (third central moment normalised by standard deviation $\sigma$) is:
\begin{eqnarray}
	\gamma_{1} &=& 0
\end{eqnarray}
Kurtosis excess (the fourth central moment normalised by standard ) is:
\begin{eqnarray}
	\gamma_{2} &=& -\frac{6}{5}
\end{eqnarray}
