\chapter{Mutual and Hedge Funds}

\section{Mutual Funds}

\subsection{Benefits}
\begin{itemize}
	\item diversification
	\item low cost of investments
\end{itemize}
One of the attractions of mutual funds for the small investor is the diversification opportunities they offer.\\
A mutual fund provides a way in which resources of many small investors are pooled so that the benefits of diversification are realized at a relatively low cost.

\subsection{Mutual Fund Types}
Money market mutual funds invest in interest-bearing instruments, such as Treasury bills, commercial paper, and bankers' acceptances, with a life of less than one year. They are an alternative to interest-bearing bank accounts and usually provide a higher rate of interest because they are not used by a government agency.\\
Money market fund investors are typically risk-averse and do not expect to lose any of the funds invested.\\
Stable value funds are a popular alternative to money market funds. They typically invest in bonds and similar instruments with lives up to five years.

\subsection{Net Asset Value}
The total number of shares outstanding goes up as investors buy more shares and down when shares are redeemed.\\
Mutual funds are valued at 4 p.m. each day.\\
This involves the mutual fund manager calculating the market value of each asset in the portfolio so that the total value of the fund is determined. This total value is divided by the number of shared outstanding to obtain the value of each share. The latter is referred to as the net asset value (NAV) of the fund.\\
Shares in the fund can be bought from the fund and sold back to the fund at any time.\\
When an investor issues instructions to buy or sell shares, it is the next calculated-NAV that applies to the transaction. 