\chapter{Banks}

\section{Bank Types}
Banks that are heavily involved in wholesale banking and may fund their lending by borrowing in financial markets are referred to as \textbf{\color{blue}money center banks}.

\section{Commercial Banking}
Among the issues addressed by regulation is
\begin{itemize}
	\item the capital that banks must keep
	\item the activities they are allowed to engage in
	\item deposit insurance
	\item and the extent to which mergers and foreign ownership are allowed
\end{itemize}

\section{Volcker rule}
Prevents proprietary trading by deposit-taking institutions.

\section{Capital Requirements}
Equity capital is characterised as "Tier 1 capital", while subordinated long-term debt is categorised as "Tier 2 capital".

\section{Bank failures in the US in 1980-1990}
The total number of failures during this decade was over 1,000 (larger than for the whole 1933 to 1979 period).

\subsection{Reasons}
There were several reasons for this:
\begin{itemize}
	\item the way banks manages interest rate risk
	\item the reduction in oil and other commodity prices, which led to many loans to oil, gas, and agricultural companies not being repaid
	\item the existence of deposit insurance allowed banks to follow risky strategies that would not otherwise be feasible. For example, they could increase their deposit base by offering high rates to depositors ans use the funds to make risky loans.
\end{itemize}

\subsection{Moral hazards}
It can be defined as the possibility that the existence of insurance changes the behaviour of the insured party.

\section{Investment Banking}

\subsection{Security Issuance}
\begin{itemize}
	\item \textbf{\color{blue}private placement} - the securities are sold to a small number of large institutional investors
	\item \textbf{\color{blue}public offering} - the securities are offered to the general public. May be on a basis of
	\begin{itemize}
		\item \textbf{\color{ForestGreen}best efforts} - the investment bank does as well as it can to place the securities with investors and is paid a fee that depends, to some extend, on its success.
		\item or \textbf{\color{ForestGreen}firm commitment} - the investment bank agrees to buy the securities from the issuer at a particular price and then attempts to sell them in the market for a slightly higher price. If for any reason it is unable to sell the securities, it ends up owning them itself.
	\end{itemize}
\end{itemize}
When equity financing is being raised and the company is already publicly traded, the investment bank can look at the prices at which the company's shares are trading a few days before the issue is to be sold as a guide to the issue price. Typically it will agree to attempt to issue new shares at a target price slightly below the current price. The main risk then is that the price of the company's shares will show a substantial decline before the new shares are sold.

\subsection{IPO}
When the company wishing to issue shares is not publicly traded, the share issue is known as an \textbf{\color{blue}initial public offer (IPO)}.\\
These types of offering are typically made on a best effort basis.\\

\subsection{Possible Problems with IPO-s}
\begin{itemize}
	\item it is difficult to determine the fair share price
	\item the bank will typically try to set the offering price below its best estimate of the market price.
	\item possible spinning
\end{itemize}

\subsubsection{Spinning}
\textbf{\color{blue}Spinning} is the practice of an investment bank offering under-priced shares of a company's initial public offering to the senior executives of a third party company in exchange for future business with the investment bank.\\
Those opposed the practise liken IPO spinning to a disguised form of a corporate bribery and believe that it cheats two classes of investors:
\begin{itemize}
	\item the shareholders in the third party company who are unable to receive similar favorable IPO terms as those received by its senior executives, and that constitutes a breach of \textbf{\color{blue}fiduciary duty} to shareholders required of the company's senior executives, specifically that they not use their corporate office to extract favours that are not shared equally by all shareholders.
	\item the \textbf{\color{blue}retail} shareholder public who are compelled to purchase large sizes of stock in an IPO at exorbitant prices from the special favoured executive friends of the brokerage underwriting the IPO. 
\end{itemize}

\paragraph{Dutch Auction Approach}
Some companies go for a Dutch Auction.\\
As for a regular IPO, a prospectus is issued and usually there is a road show. Individuals and companies bid by indicating the number of shares they want and the price they are prepared to pay. Shares are first issued to the highest bidder, and so on, until all the shares have been sold. The price paid by all successful bidders is the lowest bid that leads to a share allocation.\\
Dutch auctions potentially overcome two of the IPO problems we have mentioned:
\begin{itemize}
	\item the price that clears the market should be the market price if all potential investors have participated in the bidding process
	\item the situations where investment banks offer IPOs only to their favoured clients are avoided
\end{itemize}
However, the company does not take advantage of the relationships that investment bankers have developed with large investors that usually enable the investment bankers to sell an IPO very quickly.

\subsection{Poison Pills}
Some examples:
\begin{itemize}
	\item a potential target adds to its charter a provision where, if another company acquires one-third of the shares, other shareholders have the right to sell their shares to that company for twice the recent average share price
	\item a potential target grants to its key employees stock options that vest (i.e., can be exercised) in the event of a takeover. This is liable to create an exodus of key employees immediately, leaving an empty shell for the new owner
\end{itemize}
Poison pills should be approved by the majority of the shareholders.

\subsection{Securities Trading}

\subsection{Advisory Services}

\subsection{Potential Conflicts of Interest in Banking}

\textbf{\color{blue}Fiduciary Account} - a customer account where the bank can choose trades for the customer.

\section{Accounting}
For activities not associated with fees, there is an important distinction between the "banking book" and "trading book".\\
The \textbf{\color{blue}trading book} includes all the assets and liabilities the bank has as a result of its trading operations.\\
The \textbf{\color{blue}banking book} includes loans made to corporations and individuals. These are not marked to market. If a borrower is up-to-date on principal and interest payments on a loan, the loan is recorded in the bank's books at the principal amount owed plus accrued interest. If payments due from the borrower are more than 90 days past due, the loan is usually classified as a \textit{\color{blue}non-performing loan}. The bank does not then accrue interest on the loan when calculating its profit. When problems with the loan become more serious and it becomes likely that principal will not be repaid, the loan is classified as a \textit{\color{blue}loan loss}.\\
A bank creates a reserve for loan losses. This is a charge against the income statement for an estimate of the loan losses that will be incurred. Actual loan losses are charged against reserves.

\section{The Originate-to-Distribute Model}
\textbf{\color{blue}Originate-to-Distribute Model} involves the bank originating but not keeping loans. Portfolios of loans are packaged into tranches which are then sold to investors.\\
In the mortgage market the government sponsored agencies (the Government National Mortgage Association (GNMA - "Ginnie Mae"), the Federal National Mortgage Association (FNMA - "Fannie Mae"), and the Federal Home Loan Mortgage Corporation (FHLMC - "Freddie Mac")) buy pools of mortgages from banks and other mortgage originators, guarantee the timely repayment of interest and principal, and then package the cash flow streams and sell them to investors.\\
The investors typically take what is known as prepayment risk. This is the risk that interest rates will decrease and mortgages will be paid off earlier than expected. However, they do not take any credit risk, because the mortgages are guaranteed by GNMA, FNMA of FHLMC.\\
The originate to distribute model is also termed \textbf{\color{blue}securitisation}.

\subsection{Advantages of the Securitisation}
By securitising its loans, bank:
\begin{itemize}
	\item gets them off its balance sheet
	\item frees up funds to enable it to make more loans
	\item frees up capital that can be used to cover the risks being taken elsewhere in bank (this is particularly attractive if the bank feels that the capital required by regulators for loan is too high)
	\item bank earns a fee for originating a loan and a further fee if is services the loan after it has been sold
\end{itemize}

\section{The Risks Facing Banks}
Capital is now required for three types of risk:
\begin{itemize}
	\item credit risk
	\item market risk
	\item and operational risk
\end{itemize}


\subsection{Credit Risk}
Credit risk is the risk that counter-parties in loan and derivatives transactions will default. This has traditionally been the greatest risk facing a bank and is usually the one for which the most regulatory capital is required.

\subsection{Market Risk}
Market Risk arises primarily form the banks's trading operations. It is the risk related to the possibility that instruments in the bank's trading book will decline in value.

\subsection{Operational Risk}
Operational risk, which is often considered the biggest risk facing banks, is the risk that losses are made because internal systems fail to work as they are supposed to or because or external events.\\
The time horizon used by regulators for considering losses from credit and operational risks is one year, whereas the time horizon for considering losses from market risks is usually much shorter.\\
For example, in the case of credit and operational risk, the capital is chosen so that the chance of unexpected losses exceeding the capital in a year is 0.1\%.

\subsection{Economic Capital}
In addition to calculating regulatory capital, most large banks have systems in place for calculating what is called \textbf{\color{blue}economic capital}. Economic capital is often less than regulatory capital.\\
The form the capital can take (equity, subordinated debt, etc.) is prescribed by regulators.




