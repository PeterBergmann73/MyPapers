\chapter{Bonds}

\section{Default}
All bonds have some required ratios under the bond covenants which the debt issuer is liable to maintain. If the bond issuer does not meet any of these requirements, the issuer is considered in default.

\section{Trustee}
Trustee represents bondholders. They receive their fees from the bond issuers.

\section{Bond Maturity}
The bond maturity is the date on which the bond issuer fully repays the bond principal along with the coupon and accrued interest to satisfy its obligation under the bond indenture.\\
Sometimes bonds can be retired before they mature.

\section{Bond Types}
\begin{itemize}
	\item participating bonds - they share in the profits of the issuer or the rise in certain assets over and above the stated coupon rate
	\item income bonds - these bonds promise to pay a stipulated interest rate, but the payment is contingent on sufficient earnings. Repayment of the principal is not contingent. Interest may be cumulative or non-cumulative. If payments are cumulative, unpaid interest payments must be made up at some future date. If non-cumulative, once the interest payment is past, it does not have to be repaid. Failure to pay interest on income bonds is not an act of default and is not a cause for bankruptcy.
	\item zero coupon bonds - one important risk is eliminated - the reinvestment risk. Investors tend to find zeros less attractive in lower-interest-rate markets because compounding is not as meaningful as when rates are higher. Also, the lower the rates are, the more likely it is that they will rise again, making a zero-coupon investment worth less in the eyes fo potential holders. In bankruptcy, a zero-coupon bond creditor can claim the original offering price plus the accretion that represents accrued and unpaid interest to the date of bankruptcy filing, but not the principal amount of \$1,000. Zero-coupon bonds have been sold at deep discounts, and the liability of the issuer at maturity may be substantial. The accretion of the discount on the corporation's books is not put away in a special fund for debt retirement purposes. There are no sinking funds on most of these issues. The potentially large balloon repayment creates a cause for concern among investors.
\end{itemize}

\section{Security for Bonds}
\begin{itemize}
	\item debentures - unsecured bonds, thus the debenture bondholders have no right over the assets of the issuer
	\item mortgage bonds - grant the bondholders a first-mortgage lien on substantially all its properties
	\item collateral trust bonds - securities (stocks, notes, bonds, etc.) of other companies, that are hold by the bond issuer, are used as collateral
	\item equipment trust certificates
	\item guaranteed bonds
\end{itemize}

\section{Call and Refunding Provisions}
\begin{itemize}
	\item call provision
	\item sinking fund provision
	\item maintenance and replacement funds - provisions in indentures to maintain the property used as a collateral
	\item tender offer - is a mechanism for the early retirement of debt (bond) that is not necessarily included in the indenture of the bond. In a tender offer, the issuer of the bond sends the offering circular to the bondholders of record that present the price the issuer is willing to pay to buy back the bond and the window of time, during which bondholders can sell their bonds to the issuer.
\end{itemize}

\section{Bond Risks}
\begin{itemize}
	\item credit spread risk - is the risk of financial loss, or the under-performance of a portfolio, that arises due to movements in the credit spreads used in the marking to market of bonds
	\item credit default risk
\end{itemize}

\section{Recovery Rates}
\begin{itemize}
	\item are lower in an economic downturn
	\item are inversely correlated with default rates
	\item are not based on the size of the bond issue
	\item are higher in asset-intensive industries
\end{itemize}