\chapter{Exchanges}

\section{Exchange Functions}
An exchange performs a number of functions:
\begin{enumerate}
	\item \textbf{\color{blue}product standardisation}
	\item \textbf{\color{blue}trading venue}
	\item \textbf{\color{blue}reporting services}
\end{enumerate}


\section{Clearing}
Clearing is the term that describes the reconciling and resolving of contracts between counter-parties, and takes place between trade execution and trade settlement (when all legal obligations have been made).
The clearing is the process by which payment obligations between two or more parties are computed and netted.
\begin{itemize}
	\item direct clearing
	\item clearing rings
	\item complete/central clearing - involves a central clearing house or central counter-party that acts as the counter-party in all transactions
\end{itemize}


\section{Margining}
Margining is a method of creating a layer of security or resources to cover the losses incurred during the period of a contract. That is, margining is a process which requires members to receive and pay cash and other assets against gains and losses in their positions, which provides coverage against losses in case of default.\\
It is used both in OTC and exchange markets.

\section{Settlement}
The settlement is the process by which the contract obligations are fulfilled.

\section{Special Purpose Vehicle/Entity}
SPV (or SPE) is a separate legal entity created to isolate a firm from financial risk. The company forming SPV translates some its assets to the SPV. If a specific counterparty in a derivative transaction defaults, the firm can still receive full settlement on its other transactions without the netting losses on the defaulted transactions.\\
An SPV transforms counterparty risk into legal risk. The obvious risk is that of consolidation, which is the power of a bankruptcy court to combine the SPV assets with those of the originator.

\section{Derivative Product Companies}
DPC are typically triple A-rated independently capitalized entities created by one or more banks as a bankruptcy remote subsidiary fo a major dealer. The purpose of DPCs is to provide external counterparties with a degree of protection against counterparty risk by protecting against the default of the parent bank or parent company.\\
DPC maintain a triple-A rating by a combination of capital, margin and activity restrictions.\\
The triple-A rating of DPC typically depends on:
\begin{itemize}
	\item \text{\color{blue}minimising market risk} - DPCs can attempt to be close to market-neutral via trading offsetting contracts. Ideally, they would be on both sides of every trade as these 'mirror trades' leas to an overall matched book. Normally the mirror trades exists with the DPC parent.
	\item \text{\color{blue}support from parent}
	\item \text{\color{blue}credit risk management and operational guidelines}
\end{itemize}

\section{Monolines}
Monolines are types of insurance companies or financial guarantee companies with strong credit ratings that provide credit wraps ans credit default swaps to achieve diversification and better return.\\
They are structured as an extension of a DPC that focused only on credit default swaps.
