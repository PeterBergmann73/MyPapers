\chapter{CCP-s}

\section{Clearing Types}
Clearing can be either bilateral or central.

\section{Functions of a CCP}
CCP functions:
\begin{itemize}
	\item non-OTC trades - the primary role of CCP is to standardise and simplify operational processes.
	\item OTC CCPs - have a much more significant role to play in terms of counterparty credit risk mitigation due to the longer maturities and relative illiquidity of OTC derivatives. 
\end{itemize}

\section{Financial Markets Topology}
A CCP represents a set of rules and operational arrangements that are designed to allocate, manage and reduce counterparty risk in a bilateral market.

\section{Benefits of CCP-s}
\begin{itemize}
	\item CCP \textbf{\color{blue}reduce the interconnectedness within financial markets}, which may lessen the impact of an insolvency of a participant
	\item CCP being at the heart of trading can \textbf{\color{blue}provide more transparency on the positions of the members}
\end{itemize}
An obvious problem is that a CCP represents the centre of a system and consequently its failure would be a catastrophic event.

\section{Novation}
Contract novation - the legal process whereby the CCP is positioned between buyers and sellers.\\
Novation is the replacement of one contract with one or more other contracts.\\
Novation means that the CCP essentially steps in between parties to a transaction and therefore acts as an insurer of counterparty risk in both directions.\\
The viability of novation depends on the legal enforceability of the new contracts and the certainty that the original parties are not legally obligated to each other once the novation is completed.\\
CCP has a matched book and bears no market risk, which remains with the original party to each trade.\\
The CCP, on the other hand, does take the counterparty risk, which is centralised in the CCP structure.\\
CCP has conditional market risk.\\
In case of a default of a member, to return to a matched book, CCP can have an auction of the defaulting member's positions.\\
CCP also mitigate counterparty risk by demanding financial resources form their members that are intended to cover the potential losses in the event that one or more of them default.

\section{Advantages of Central Clearing}
\begin{itemize}
	\item multilateral offset - in relation to various aspects such as cashflows or margin requirements
	\item the CCP's functions like margining, netting, and settlement potentially increase operational efficiency and reduce costs
	\item the central auction feature of CCP may transform the large default of a clearing member into smaller price disruptions through coordinated replacement of positions during a crisis
\end{itemize}

\section{Disadvantages of CCP}
\begin{itemize}
	\item moral hazard - effect on the counterparty risk management practices of the market participant as they believe that, in presence of the CCP, they do not have to take risk into consideration
	\item the function of frequently requiring greater margin requirements under a CCP may increase procyclicality in the economy
	\item the central counterparty is vulnerable to adverse selection, which means that since the members trading OTC derivatives know more about the risks than the CCP itself, the members may intentionally pass the toxic contracts or assets to the CCP
\end{itemize}

\section{CCP Properties}
\begin{itemize}
	\item the counterparty risk does not disappear from the system, but it is transferred from one place to another through CCPs
	\item CCPs are also vulnerable to failure
	\item the margining function of CCPs is a double-edged sword - it increases the market liquidity and decreases the liquidity position of the participants
\end{itemize}

\section{CCP Risks}
\begin{itemize}
	\item default-related risk:
	\begin{itemize}
		\item distress of clearing members - if a clearing member defaults, other CCP members might experience distress due to default correlation. Default events are unlikely to be isolated events.
		\item failed auction
		\item reputational risk - it arises after a CCP allocates losses to other clearing members after the default of a member
		\item resignation of the clearing members
	\end{itemize}
	\item non-default related risk
	\begin{itemize}
		\item loss on the investments
		\item fraud
		\item liquidity risk
		\begin{itemize}
			\item due to large amount of cash that flows through the CCPs due to variation margin requirements
			\item due to Basel III leverage ratio requirements - the required leverage rate is defined as a bank's tier one capital divided by its exposure, which should be at least 3\%
		\end{itemize}
		\item sovereign risk - faced by CCP due to the failure of members who have sovereign bonds as margin, which may have declined in value due to sovereign failure
		\item custodian risk - banks and financial institutions act as the custodians for CCPs. However, these banks can face technical or human failures that could stop a bank from processing the CCP's instructions to make cash payments and receive and deliver securities to its members. This could create not only liquidity problems on the end of CCP and members, but can also increase the systematic risk in the market.	
	\end{itemize}
\end{itemize}

\section{OTC CCPs}
OTC CCPs have to deal with complex transactions and projects.\\
These CCPs are exposed to the risk of inconsistency in its margining functions.\\
The CCP is most exposed to model risk in modelling of the initial margin requirements as compared to in the modelling of the variation margin requirements. This is due to considering misspecification in the model.\\
Model-based initial margin estimations curry the risk that, the initial margins increases in proportion to the size of the position without considering that the risk of a large and concentrated position is adequately covered.

\section{Margining}
The margin requirements are solely dependent on the risk of the transaction. The CCP evaluates the risk of the transaction and, based on the risk of the transaction, requires a certain margin from the parties involved.\\
The initial margin is the additional amount required from members at the inception of the trade; it provides coverage in case one member defaults.\\
The variation margin requires members to pay or receive the cash or other securities against gains or losses in their positions during the contract.

\section{Loss Mutualisation}
All CCP members contribute to the default fund.\\
The process of insurance and absorbing the losses of a defaulting counterparty through a specific pool is called loss mutualisation.\\
In other words, loss mutualisation is the process in which all the CCP members contribute a specific amount of resources to a pool which is used to absorb the losses if the resources included in the initial margin, variation margin, and default fund contribution of the defaulting counterparty are insufficient to pay off the losses.

\section{Need for Multiple CCP-s}
\begin{itemize}
	\item it is easy to clear transactions for a regional CCP to clear transactions with locally denominated contracts/currencies
	\item it is difficult for one global CCP to specialise in every product being traded in each global exchange
\end{itemize}


\section{Revenue Streams for CCP}
\begin{itemize}
	\item the fees charged per trade on the members for the clearing services
	\item and the interests that they earn on the margins posted by the members
\end{itemize}
