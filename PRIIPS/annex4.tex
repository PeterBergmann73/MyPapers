\chapter{Performance Scenarios}

\section{Summary}
The performance scenarios shall be the following:
\begin{enumerate}
	\label{Sec:PerformanceScenarios}
	\item a favourable scenario
	\item a moderate scenario
	\item an unfavourable scenario
	\item a stress scenario
\end{enumerate}
The stress scenario shall show intermediate periods where those periods would be shown for the performance scenarios (1-3) above (\ref{Sec:PerformanceScenarios})

\section{Scenario Calculations}
The scenario shall be calculated in a similar manner as the market risk measure.\\
The scenario values shall be calculated \textbf{\color{blue}for the recommended holding period}.

\section{Scenario Percentiles}
The following percentiles should be used for scenarios:
\begin{center}
	\begin{longtable}{| c | c |}
		\hline
		\textbf{Scenario} & \textbf{Percentile}\\
		\hline
		unfavourable & 10-th\\
		\hline
		moderate & 50-th\\
		\hline
		favourable & 90-th\\
		\hline
	\end{longtable}
\end{center}

\section{Stress Scenario}
\label{Sec:StressScenario}

\subsection{Summary}
The stress scenario shall be the value of the PRIIP that results from the methodology outlined in points ...\\

\subsection{Stress Volatility}
The following steps are required to calculate the stress volatility:
\begin{enumerate}
	\item identify a sub interval of length $\omega$ which corresponds to the following intervals:
	\begin{center}
		\begin{longtable}{| c | c | c|}
			\hline
			\textbf{Observation Frequency} & $<=$\textbf{ 1 year} & $> $ \textbf{ 1 year}\\
			\hline
			Daily prices & 21 & 63\\
			\hline
			Weekly prices & 5 & 16\\
			\hline
			Monthly prices & 6 & 12\\
			\hline
		\end{longtable}
	\end{center}
	\item identify for each sub interval of length $\omega$ the historical lognormal returns $r_{t}$,\\
	where:\\
	$t = t_{0}$, $t_{1}$, $t_{2}$, ..., $t_{N}$ 
	\item measure the volatility based on the formula below starting from $t_{i} = t_{0}$ rolling until $t_{i} = t_{N - \omega}$
\end{enumerate}

\subsection{For Category 3 PRIIPs}
\label{Sec:StressScenarioForCategory3}

\section{Calculation of Expected Values for Intermediate Holding Periods}

\subsection{Intermediate Periods}
\begin{center}
	\begin{tabular}{| c | p{10cm} |}
		\hline
		\textbf{RHP} & \textbf{Periods to show performance}\\
		\hline
		$<=$ 1 year & no intermediate holding periods\\
		\hline
		\multirow{2}{*}{between 1 and 3 years} & - at the end of the first year\\
		& - and at the end of the RHP\\
		\hline
		\multirow{3}{*}{$>= $ 3 year} & - at the end of the first year\\
		& - after half the RHP rounded up to the end of the nearest year\\
		& - and at the end of RHP\\
		\hline
	\end{tabular}
\end{center}

\subsection{Category 2}

\subsection{Category 3}
To produce the favourable, moderate, unfavourable and stress scenarios at an intermediate period before the end of the recommended holding period, the manufacturer shall pick three underlying simulations as referred to in (\ref{Sec:MrmCalculationForCategory3}) for the calculation of MRM and one underlying simulation as referred in (\ref{Sec:StressScenarioForCategory3}), on the basis of underlying levels only and in such a manner that the simulated value of the PRIIPs for that intermediate period is likely to be consistent with the relevant scenario.\\
The manufacturer shall choose underlying values consistent with the 90-th, the 50-th, and the 10-th percentile levels and the percentile level that correspond to 1\% for 1 year an to 5\% for the other holding periods of the PRIIP and use these values as the seed values for a simulation to determine the value of the PRIIP.

\subsection{Common Features}
For \textbf{\color{blue}favourable, moderate and unfavourable scenarios at intermediate periods}, the estimate of the distribution used to read the value of the PRIIP at different percentiles shall be consistent with the observed return and volatility observed over the past 5 years of all market instruments that determine the PRIIP's value.\\
For \textbf{\color{blue}the stress scenario at intermediate periods}, the estimate of the distribution used to read the value of the PRIIP at different percentiles shall be consistent with the simulated distribution of all market instruments that determine the PRIIP's value as set out in (\ref{Sec:StressScenario}).


 


