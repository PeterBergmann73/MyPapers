\chapter{Performance Scenarios}

\section{Summary}
The performance scenarios shall be the following:
\begin{enumerate}
	\label{Sec:PerformanceScenarios}
	\item a favourable scenario
	\item a moderate scenario
	\item an unfavourable scenario
	\item a stress scenario
\end{enumerate}
The stress scenario shall show intermediate periods where those periods would be shown for the performance scenarios (1-3) above (\ref{Sec:PerformanceScenarios})

\section{Scenario Calculations}
The scenario shall be calculated in a similar manner as the market risk measure.\\
The scenario values shall be calculated \textbf{\color{blue}for the recommended holding period}.

\section{Scenario Percentiles}
The following percentiles should be used for scenarios:
\begin{center}
	\begin{longtable}{| c | c |}
		\hline
		\textbf{Scenario} & \textbf{Percentile}\\
		\hline
		unfavourable & 10-th\\
		\hline
		moderate & 50-th\\
		\hline
		favourable & 90-th\\
		\hline
	\end{longtable}
\end{center}

\section{Stress Scenario}

\subsection{Summary}
The stress scenario shall be the value of the PRIIP that results from the methodology outlined in points ...\\

\subsection{Stress Volatility}
The following steps are required to calculate the stress volatility:
\begin{enumerate}
	\item identify a sub interval of length $\omega$ which corresponds to the following intervals:
	\begin{center}
		\begin{longtable}{| c | c | c|}
			\hline
			\textbf{Observation Frequency} & $<=$\textbf{ 1 year} & $> $ \textbf{ 1 year}\\
			\hline
			Daily prices & 21 & 63\\
			\hline
			Weekly prices & 5 & 16\\
			\hline
			Monthly prices & 6 & 12\\
			\hline
		\end{longtable}
	\end{center}
	\item identify for each sub interval of length $\omega$ the historical lognormal returns $r_{t}$,\\
	where:\\
	$t = t_{0}$, $t_{1}$, $t_{2}$, ..., $t_{N}$ 
	\item measure the volatility based on the formula below starting from $t_{i} = t_{0}$ rolling until $t_{i} = t_{N - \omega}$
\end{enumerate}

\subsection{For Category 3 PRIIPs} 


