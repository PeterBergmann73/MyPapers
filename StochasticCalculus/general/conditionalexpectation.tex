\chapter{Conditional expectations}

TODO - Joshi1.page155 - creation of martingales.

\section{Properties}
\begin{enumerate}
	\item Conditional expectation based on \textbf{\color{blue}no information} is the ordinary expectation:
	\begin{eqnarray}
	\mathbb{E}(X\vert \mathscr{F}_{0}) &=& \mathbb{E}(X)
	\end{eqnarray}
	\item \textbf{\color{blue}The Tower Law} - if $s < t$ and we first take the conditional expectation at time $t$ followed by the conditional expectation at time $s$, then this is the same as taking the conditional expectation at time $s$:
	\begin{eqnarray}
		\mathbb{E}(\mathbb{E}(X\vert \mathscr{F}_{t})\vert \mathscr{F}_{s}) &=& \mathbb{E}(X \vert \mathscr{F}_{s})
	\end{eqnarray}
	\item If we \textbf{\color{blue}condition on information that is independent of the value of the random variable} then we get the same value as conditioning on no information.
	\paragraph{Random variable independent of the information.}
	If changing the path up to time $s$ does not affect the value of the random variable, then the random variable is independent of $\mathscr{F}_{s}$.\\
	So if $X$ is independent, we have:
	\begin{eqnarray}
		\mathbb{E}(X\vert \mathscr{F}_{s}) &=& \mathbb{E}(X)
	\end{eqnarray}
	\item If \textbf{\color{blue}the random variable is determined by the information in $\mathscr{F}_{s}$}, then conditioning on that information will have no effect, and therefore:
	\begin{eqnarray}
	\mathbb{E}(X\vert\mathscr{F}_{s}) &=& X
	\end{eqnarray}
\end{enumerate}

