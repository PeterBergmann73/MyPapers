\subsection{Radon-Nikodym Derivative}

\subsubsection{Radon-Nikodym Theorem}
\textit{If a {\color{blue}probability measure} $\mathbb{Q}$ is {\color{blue}absolutely continuous (equivalent)} with respect to a probability measure $\mathbb{P}$, then it can be written as:}
\begin{equation}
	\mathbb{Q} = \int_{E}{fd\mathbb{P}}
\end{equation}
\subsubsection{Radon-Nikodym Derivative}
By analogy with the first fundamental theorem of calculus, the function $f$ is called the Radon-Nikodym derivative of $\mathbb{Q}$ with respect to $\mathbb{P}$:
\begin{eqnarray}
f = \frac{d\mathbb{Q}}{d\mathbb{P}}
\end{eqnarray}
A measure change consists of re-weighting the probability of paths. We therefore construct them by multiplying probabilities by a random variable.