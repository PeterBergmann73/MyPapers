\subsection{Black-Scholes formula derivation}

\subsubsection{Steps}
\begin{itemize}
	\item Write a value of an option as a discounted payoff expectation at maturity
	\item Write the distribution of the stock price at maturity
	\item Put it into the discounted payoff at maturity
	\item Rewrite the discounted payoff at maturiy using LOTUS (\textbf{\color{red} note $x$ appearing in the integral instead of the normal distribution $N(0, 1)$})
	\item Change the integration bounds to get rid off the max function under the integral
	\item Solve the integral bounds for $x$
	\item Denote the right-hand side of the inequality as $l$
	\item Split the integral into two parts - one with $x$ and with strike $K$
	\item Note that the second part is just a normal CDF from $l$ to $\infty$. Find its value as $N(-l)$ using evenness of CDF and replacing the integral $\int_{l}^{\infty}$ with $\int_{-\infty}^{-l}$
	\item Solve the first integral part by making the part in the exponent a full square
\end{itemize}


\subsection{Derivation}
\begin{itemize}
	\item \textbf{\textit{Write a value of an option as a discounted payoff expectation at maturity:}}\\
A price of a call option at time $0$ is defined as:
\begin{eqnarray}
C &=& e^{-rT}\mathbb{E}((S_{T} - K)_{+})
\end{eqnarray}
	\item \textbf{\textit{Write the distribution of the stock price at maturity}}:\\
In the risk-neutral world we have that:
\begin{eqnarray}
S_{t} &=& S_{0}exp\bigg(rT - \frac{1}{2}\frac{\sigma ^{2}}{2}T + \sigma\sqrt{T}N(0, 1)\bigg)
\end{eqnarray}
	\item \textbf{\textit{Put it into the discounted payoff at maturity}}:\\
	The value of our option is:
	\begin{eqnarray}
	\frac{B_{0}}{B_{T}}\mathbb{E}\bigg[\bigg(S_{0}exp\bigg(rT - \frac{1}{2}\frac{\sigma ^{2}}{2}T + \sigma\sqrt{T}N(0, 1)\bigg) - K\bigg)_{+}\bigg]
	\end{eqnarray}
	\item \textbf{\textit{Rewrite the discounted payoff at maturiy using LOTUS}}:\\
	Recalling that the density of $N(0, 1)$ is $\frac{1}{\sqrt{2\pi}}e^{-x^{2}/2}$, we can write this as:
	\begin{eqnarray}
		\frac{e^{-rT}}{\sqrt{2\pi}}\int{e^{-\frac{x^{2}}{2}}\bigg(S_{0}exp\bigg(rT - \frac{1}{2}\frac{\sigma ^{2}}{2}T + \sigma\sqrt{T}x\bigg) - K\bigg)_{+}}dx
	\end{eqnarray}
	\item \textbf{\textit{Change the integration bounds to get rid off the max function under the integral}}:\\
	The integral is non-zero if and only if:
	\begin{eqnarray}
		S_{0}exp\bigg(rT - \frac{1}{2}\frac{\sigma ^{2}}{2}T + \sigma\sqrt{T}x\bigg) \geq K
	\end{eqnarray}
	\item \textbf{\textit{Solve the integral bounds for $x$:}}\\
	Thus the integral must be taken over:
	\begin{eqnarray}
	x \geq \frac{\log{K/S_{0}} + \frac{1}{2}\sigma^{2}T - rT}{\sigma\sqrt{T}}
	\end{eqnarray}
	\item \textbf{\textit{Denote the right-hand side of the inequality as $l$}}
	\item \textbf{\textit{Split the integral into two parts - one with $x$ and with strike $K$}}:\\
	Our integral now has two terms. The second simple term is just:
	\begin{eqnarray}
		\frac{e^{-rT}}{\sqrt{2\pi}}\int_{l}^{\infty}{e^{-\frac{x^{2}}{2}}Kdx}
	\end{eqnarray}
	\item \textbf{\textit{Note that the second part is just a normal CDF from $l$ to $\infty$. Find its value as $N(-l)$ using evenness of CDF and replacing the integral $\int_{l}^{\infty}$ with $\int_{-\infty}^{-l}$}}:\\
	The second term is therefore equal to:
	\begin{eqnarray}
		e^{-rT}KN\Bigg(\frac{\log{S_{0}/K}+rT-\frac{1}{2}\sigma^{2}T}{\sigma\sqrt{T}}\Bigg)
	\end{eqnarray}
	\item \textbf{\textit{Solve the first integral part by making the part in the exponent a full square}}:\\
	Changing the variables:
	\begin{eqnarray}
	x &=& \bar x + \sigma\sqrt{T}
	\end{eqnarray}
	we get:
	\begin{eqnarray}
	\frac{e^{-rT}}{\sqrt{T}}\int_{l - \sigma\sqrt{T}}^{\infty}{e^{-\frac{\bar x^{2}}{2}}S_{0}e^{rT}d\bar{x}}
	\end{eqnarray}
\end{itemize}